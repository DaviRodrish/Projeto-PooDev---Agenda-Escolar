\documentclass[12pt,a4paper]{report}

% ------------------------------
% Pacotes básicos
% ------------------------------

\usepackage{mathptmx}
\usepackage[utf8]{inputenc}
\usepackage[T1]{fontenc}
\usepackage[brazil]{babel}
\usepackage{geometry}
\usepackage{setspace}
\usepackage{graphicx}
\usepackage{hyperref}
\usepackage{longtable}
\usepackage{array}
\usepackage{float}
\usepackage{caption}

% Layout
\geometry{margin=2.5cm}
\onehalfspacing

% Configuração dos links
\hypersetup{
    colorlinks=true,
    linkcolor=blue,
    urlcolor=blue
}

\begin{document}

% ------------------------------
% CAPA
% ------------------------------
\begin{titlepage}
    \begin{center}
        
        \includegraphics[scale=0.2]{imagens/uenf1.png}\\[1cm]
       
        {\large CENTRO DE CIÊNCIAS E TECNOLOGIAS}\\
        {\large LABORATÓRIO DE CIÊNCIAS MATEMÁTICAS}\\[4cm]

        {\large \textbf{Davi Rodrigues Soares Machado}}\\[4cm]

        {\large \textbf{Relatório: \\ Sistema de Rede Escolar}}\\[4cm]

        Campos dos Goytacazes - RJ\\
        \today
    \end{center}
\end{titlepage}


\tableofcontents
\newpage


% ------------------------------
\chapter{Arquitetura Final do Sistema}
% ------------------------------

\section{Visão Geral da Arquitetura}
A arquitetura final do sistema foi planejada seguindo o modelo em camadas, integrando a interface desenvolvida com React/Next.js com um backend estruturado e conectado a um banco de dados PostgreSQL.

O objetivo principal era permitir o gerenciamento das informações escolares de alunos, professores e demais usuários dentro da Rede Escolar.

A comunicação entre backend e frontend foi realizada usando API REST.

\section{Componentes Principais}
\subsection{Frontend}
O principal componente do frontend é o next, que serviu para renderizar e otimizar as rotas entre as pages. Já o react, permitiu a criação de componentes reutilizáveis do sistema
\paragraph{Principais Páginas e Módulos}
A ideia principal consistia em criar uma home page central, com informações da rede de escolas, e uma pagina destinada a cada tipo de usuário (alunos, professores e secretários). 

\subsection{Backend}
O backend foi desenvolvido com Python e FastAPI, sendo o local responsável pela parte lógica do sistema, o tratamento de autenticação de usuário e a consistencia dos dados que iriam para o banco de dados.

As requisições vindas dos usuários pelo frontend eram processadas e direcionadas de volta através de respostas em JSON

\subsection{Banco de dados}
O modelo inicial do banco de dados consistia em uma classe geradora Usuário, e suas subclasses como Alunos, Professores e Secretários. No decorrer do projeto, foi visto a necessidade de criar tabelas direcionadas as notas dos alunos.

\section{Considerações}
O andamento do desenvolvimento sofreu mudanças, que serão abordadas na Sessão Resumo Mensal, sendo comparado as atividades feitas com as atividades propostas.

% ------------------------------
\chapter{Cronograma Efetivamente Cumprido}
% ------------------------------

\section{Tabela de Cronograma}
\begin{figure}[H]
\centering
\includegraphics[width = \linewidth]{imagens/Calendario1bi.png}
\caption{Cronograma do Primeiro Bimestre de Desenvolvimento}
\label{Cronograma do Primeiro Bimestre de Desenvolvimento}
\end{figure}

\begin{figure}[H]
\centering
\includegraphics[width = \linewidth]{imagens/Calendario2bi.png}
\caption{Cronograma do Segundo Bimestre de Desenvolvimento}
\label{Cronograma do Segundo Bimestre de Desenvolvimento}
\end{figure}

\section{Resumo Mensal}
Toda a comparação vai se seguir utilizando do github, onde foi registrado cada uma das alterações.

\begin{itemize}
\item \textbf{SETEMBRO:} No mês de Setembro, o principal foco era elucidar o que seria em si o sistema, quais as funcionalidades e conferir quais os requisitos que seriam essenciais. Na primeira semana, tudo ocorreu como o planejado no cronograma. Houve a primeira reunião e a apresentação do que consistia o projeto, com orçamento e tecnologias que seriam utilizadas. Nesse primeiro mês, o foco principal aconteceu na documentação que precisou passar por mudanças, deixando a quinzena final do mês para a prototipagem do que seriam as classes principais do sistema.

\item \textbf{OUTUBRO:} O Mês de outubro, segundo o cronograma seria para a implementação do backend e ao final, ter mais uma reunião para retificar o andamento do projeto. No decorrer do mês, houveram problemas pessoais que fizeram com que o desenvolvimento do backend fosse drasticamente afetado, atrasando a entrega do mesmo em 3 semanas. 

Tendo uma mudança significativa somente ao fim do mês, onde o início do frontend começou a ser desenvolvido, na esperança de retomar ao cronograma,algumas dificuldades foram aparecendo. O resultado por trás do atraso no backend foi postergar o desenvolvimento dos principais métodos e modulos das classes. A decisão de migrar pro frontend e começar um rascunho do que seria o projeto final fez com que o desenvolvimento dos módulos fosse deixado em segundo plano. Por fim, a reunião que deveria acontecer ao final do mês não aconteceu, deixando ainda mais o desenvolvimento do sistema, distante do requerido pelo cliente.

\item \textbf{NOVEMBRO:}Novembro tinha como objetivo terminar o frontend e começar os testes funcionais do sistema, na ultima semana. Novamente, o desenvolvimento foi prejudicado, dessa vez em vista das provas e de entregas de trabalhos, principalmente matérias com conteúdos pesados, como pesquisa operacional e compiladores. No decorrer do mes de novembro, integramos o render no projeto, ferramenta que estabiliza e conecta a comunicação entre back e frontend.

Os principais commits desse mes foram voltados para a parte de credenciais de usuários, mas a comunicação entre o banco de dados estava sendo quebrada dado ao mau uso da API de comunicação. A alternativa foi voltar a comunicação em um ambiente interno virtual, onde atualmente acontece a validação das credenciais dos usuários.

A reunião prevista com o cliente não aconteceu da forma planejada, sendo demonstrado o estado atual do sistema somente no dia 26 de novembro, onde pude ver que colocar as funções e módulos do sistema em segundo plano, deixasse o sistema somente no esboço, dado que o foco principal foi colocado de lado. As Páginas dos usuários foram visualmente feitas, mas como os módulos nao estavam implementados, ficaram sem funcionalidades.

\item \textbf{DEZEMBRO:} Dezembro seria o mês em que o projeto seria entregue completo, sendo o dia 03 o dia final da apresentação do sistema. Como dito anteriormente, o sistema ainda não esta finalizado e depende dessa ultima semana, que seria antes para correção de bugs e testes de uso, mudando para novamente os primórdios do backend: os módulos e funções de cada classe. Essencialmente o sistema é oco, visualmente completo mas internamente vazio. A pretenção é fazer o máximo proposto ate o dia final, para que as expectativas do cliente sejam minimamente supridas.
 
\end{itemize}

% ------------------------------
\chapter{Descrição de Desafios Encontrados}

\begin{itemize}
    \item Integração entre Backend,Frontend e Banco de dados: Acredito que pela falta de experiência, uma das maiores dificuldades foi a comunicação entre esses 3 componentes. A solução foi pesquisar ferramentas que permitiriam facilitar essa comunicação. Primariamente, escolhi usar o render mas não tive total familiaridade, então optei por fazer um ambiente virtual usando python, que através do uvicorn, consigo gerar uma comunicação local sem muitos problemas.
    \item Dificuldades com tempo de desenvolvimento: Sem dúvidas, o tempo foi o maior fator dificultador. O cronograma foi feito baseado nas semanas, levando em consideração entregas e provas de outras disciplinas, mas mesmo com isso, não foi como previsto. O acúmulo de tarefas que foram postergadas acabou impedindo que o projeto andasse da forma com que foi previsto. O resultado foi o não cumprimento do cronograma e o atrado do desenvolvimento dos módulos.
    \item Conhecimento prévio: A construção do sistema não levava em conta o desconhecimento ou a falta de experiência em tecnolgias como o github, a programação orientada à objetos, que além de precisar revisar os conceitos e usos, nunca tinha sido feita por mim na linguagem python. Acredito que boas horas foram dedicadas ao estudo e preparo antes mesmo do desenvolvimento, para a familiarização dessas tecnologias.
    \item Má elaboração do banco de dados: Ao criar o banco de dados, percebi que o diagrama feito não levava em consideração o armazenamento de dados que seriam usados dentro das funções e módulos, como notas dos alunos e como isso seria projetado. Ainda estamos vendo a viabilidade da implementação no sistema e das alterações nos diagramas.
\end{itemize}
% ------------------------------
\chapter{Conclusão}
O desenvolvimento, mesmo que parcial, do sistema foi e tem sido um bom desafio. Unir o conhecimento de várias disciplinas da faculdade, que por vezes não temos o tempo de por em prática cria uma bagagem que futuramente acredito ser essencial para o desenvolvimento de futuros sistemas. 

Em grande parte da faculdade vemos o desenvolvimento dos sistemas baseado em metologias pouco práticas no dia a dia, como o desenvolvimento cascata. Ná prática, essa metodologia por mais segura que seja, impede que o desenvolvimento dinâmico aconteça, tornando tarefas que poderiam ser feitas em horas, levarem dias, dado a necessidade da espera da etapa anterior ter sido concluída.

O estudo dos requisitos de sistema foi crucial para que a ideia que o cliente tinha fosse estruturada da forma mais real, colocando ambos os lados (cliente e desenvolvedor) em consentimento. 

Por fim,acredito que essa tenha sido uma experiência marcante na universidade, definitivamente a complexidade de um sistema vai muito alem do que imaginamos, e a necessidade de novos componentes vem de acordo com a demanda do cliente, mostrando a importancia que a comunicação entre ambos os lados tem para o desenvolvimento. 

\end{document}
